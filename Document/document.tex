% Θεμελιώσεις Κρυπτογραφίας 2016
% Εργασία #2
% Κωσταντίνος Σαΐτας - Ζαρκιάς - 2406
% Οδυσσεύς Κρυσταλάκος - 2362
%-------------------------------------------------------------------------

\documentclass[a4paper, 11pt]{article}


\usepackage[english,greek]{babel} % the last language is the default
	\usepackage[utf8x]{inputenc}

%% > UNCOMMENT if your editor uses iso-8859-7 encoding for Greek (typical in Windows System).
% \usepackage[iso-8859-7]{inputenc}

\usepackage{enumerate}
\usepackage{seqsplit}
\usepackage{hyperref}
\usepackage[pdftex]{graphicx}


\newcommand{\lt}{\latintext}
\newcommand{\gt}{\greektext}
\newcommand\tab[1][1cm]{\hspace*{#1}}
%-------------------------------------------------------------------------

\title{Εργασία 2}

\author{Κωσταντίνος Σαΐτας - Ζαρκιάς - 2406 \\ Οδυσσεύς Κρυσταλάκος - 2362}

\date{\today}

%--------------------------------------------------------------------------
\begin{document}

\maketitle

% ===== Θέμα 1 =====
\section*{Θέμα 1}


\newpage


% ===== Θέμα 2 =====
\section*{Θέμα 2}
\subsection*{({\lt i})}
Υλοποιώντας και χρησιμοποιώντας τον εκτεταμένο αλγόριθμο του Ευκλείδη για την εύρεση του {\lt GCD}, βρέθηκε:

\[GCD(126048, 5050) = 202\]
\[-1 \cdot 126048 + 25 \cdot 5050 = 202\]

\subsection*{({\lt ii})}
Για τον υπολογισμό του αντίστροφου, υπολογίσθηκαν όλα τα γινόμενα $ 809 * i $ όπου $i$ παίρνει τιμές από 1 έως 1000. Βρέθηκε πως ο αντίστροφος είναι το 464.

\subsection*{({\lt iii})}
Καθώς το $2^{100}$ είναι δύσκολο να αποθηκευτεί και να χρησιμοποιηθεί σε πράξεις με ακρίβεια, χρησιμοποιήθηκε μία διαφορετική τεχνική. Υπολογίσθηκε το 2 {\lt modulo} 101 και το αποτέλεσμα πολλαπλασιάστηκε({\lt modulo} 101) με το 2. Αυτό έγινε επαναληπτικά 100 φορές και το αποτέλεσμα είναι 464.

\subsection*{({\lt iv})}
Ο αλγόριθμος υλοποιήθηκε στο αρχείο {\lt fast.py}. Τα αποτελέσματα είναι:
\[2^{1234567} \textrm{\lt mod} 12345 = 8648\]
\[130^{7654321} \textrm{\lt mod} 567 = 319\]
\newpage


% ===== Θέμα 3 =====
\section*{Θέμα 3}

\newpage


% ===== Θέμα 4 =====
\section*{Θέμα 4}

% ===== Θέμα 5 =====
\section*{Θέμα 5}


\newpage
% ===== Θέμα 6 =====
\section*{Θέμα 6}

\newpage


% ===== Θέμα 7 =====
\section*{Θέμα 7}

% ===== Θέμα 8 =====
\section*{Θέμα 8}

\newpage


% ===== Θέμα 9 =====
\section*{Θέμα 9}
\subsection*{{\lt (i)}}

\end{document}
