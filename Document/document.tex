% Θεμελιώσεις Κρυπτογραφίας 2016
% Εργασία #2
% Κωσταντίνος Σαΐτας - Ζαρκιάς - 2406
% Οδυσσεύς Κρυσταλάκος - 2362
%-------------------------------------------------------------------------

\documentclass[a4paper, 11pt]{article}


\usepackage[english,greek]{babel} % the last language is the default
	\usepackage[utf8x]{inputenc}

%% > UNCOMMENT if your editor uses iso-8859-7 encoding for Greek (typical in Windows System).
% \usepackage[iso-8859-7]{inputenc}

\usepackage{enumerate}
\usepackage{seqsplit}
\usepackage{hyperref}
\usepackage[pdftex]{graphicx}


\newcommand{\lt}{\latintext}
\newcommand{\gt}{\greektext}
\newcommand\tab[1][1cm]{\hspace*{#1}}
%-------------------------------------------------------------------------

\title{Εργασία 2}

\author{Κωσταντίνος Σαΐτας - Ζαρκιάς - 2406 \\ Οδυσσεύς Κρυσταλάκος - 2362}

\date{\today}

%--------------------------------------------------------------------------
\begin{document}

\maketitle

% ===== Θέμα 1 =====
\section*{Θέμα 1}
\subsection*{({\lt vi})}
Υπάρχουν 4 τύποι επιθέσεων στις ψηφιακές υπογραφές:
\begin{itemize}
	\item {\lt Existential forgery}: Ο επιτιθέμενος μπορεί να παράξει υπογραφεί για κάποιο μήνυμα $m$ στο οποίο δεν έχει καμία επιρροή. Αυτό σημαίνει ότι ο επιτηθέμενος δεν διαλέγει το $m$ και το μήνυμα δεν είναι απαραίτητο να έχει κάποιο νόημα.
	\item {\lt Selective forgery}: Ο επιτηθέμενος επιλέγει ένα μήνυμα $m$ και στην συνέχεια μπορεί να παράξει υπογραφή για αυτό το συγκεκριμένο $m$.
	\item {\lt Universal forgery}: Ο επιτηθέμενος μπορεί να παράξει ψηφιακή υπογραφή για οποιοδήποτε μήνμυμα $m$
	\item {\lt Total break}: Ο επιτηθέμενος αποκτά πρόσβαση στο ιδιωτικό κλειδί.
\end{itemize}

\subsection*{({\lt vii})}
Με $p$ και $q$ γνωστά μπορούμε να υπολογίσουμε το $N$ και $φ(N)$.
\[N = p \cdot q = 463 \cdot 547 = 253261\]
\[\phi(N) = (p-1)\cdot(q-1) = 462 \cdot 546 = 252252\]

επομένως το $d$ υπολογίζεται:
\[ed\equiv1(\textrm{\lt mod} \phi(N)) \Leftrightarrow d = 27473 \]

Τελικά για την αποκρυπτογράφηση του c αρκεί να υπολογισθεί
\[ c^d (\textrm{\lt mod} N) = 12584\]


\subsection*{({\lt ix})}
Έστω ότι έχουμε σύστημα ψηφιακής υπογραφής με τα εξής στοιχεία: \\
$H$: Συνάρτηση κατακερματισμού\\
$p$: Μεγάλος πρώτος ακέραιος\\
$g$: Τυχαίος {\lt generator} στην κυκλική ομάδα $Z^*_p$\\
$x$: Μυστικό κλειδί $< p$\\
$y$: Δημόσιο κλειδί με $y=g^x\textrm{\lt mod}p}$\\

Έστω πώς η {\lt Alice} θέλει να υπογράψει ένα μήνυμα $m$ και να το στείλει στον {\lt Bob}. Κατά τη διαδικασία της υπογραφής, επιλέγεται ένα τυχαίο $k$ έτσι ώστε $ 1 < k < p-1 $. Επίσης υπολογίζονται:
\[ r \equiv g^k (\textrm{\lt mod} p) \]
\[ s \equiv k^{-1}(H(m)-xr) (\textrm{\lt mod} p-1)\]
Έτσι στον {\lt Bob} αποστέλονται τα $(m,r,s)$. Φυσικά αυτά μπορεί να τα λάβει και οποιοσδήποτε άλλος παρακολουθεί το κανάλι επικοινωνίας.

Έστω ότι η {\lt Alice} χρησιμοποιεί δύο φορές το ίδιο $k$. Τότε για δύο διαφορετικά μηνύματα $m_1$ και $m_2$ ισχύει:
\[ s_1 \equiv k^{-1}(H(m_1)-xr) (\textrm{\lt mod} p-1)\]
\[ s_2 \equiv k^{-1}(H(m_2)-xr) (\textrm{\lt mod} p-1)\]
εκ των οποίων προκύπτει:
\[k(s_2 - s_1) = H(m_2) - H(m_1)\]
Από αυτή την εξίσωση μπορούν να βρεθούν μία σειρά από $k$ τα οποία αν αντικατασταθούν στην σχέση:
\[ r \equiv g^k (\textrm{\lt mod} p) \]
μπορεί να βρεθεί η μοναδική σωστή τιμή του $k$. Με αντικατάσταση στις παραπάνω σχέσεις, μπορεί να βρεθεί το ιδιωτικό κλειδί.

Τελικά, o {\lt Bob} ή οποιοσδήποτε άλλος παρακολουθεί το κανάλι μπορεί να υπογράφει με την υπογραφή της {\lt Alice}.

\subsection*{({\lt x})}
Ισχύει
\[\phi(n) = (p-1)(q-1) = pq - p - q + 1 = pq - (p+q) + 1\]
και επειδή $N = pq$
\[\phi(n) = N - (p+q) + 1\]
\newpage


% ===== Θέμα 2 =====
\section*{Θέμα 2}
\subsection*{({\lt i})}
Υλοποιώντας και χρησιμοποιώντας τον εκτεταμένο αλγόριθμο του Ευκλείδη για την εύρεση του {\lt GCD}, βρέθηκε:

\[GCD(126048, 5050) = 202\]
\[-1 \cdot 126048 + 25 \cdot 5050 = 202\]

\subsection*{({\lt ii})}
Για τον υπολογισμό του αντίστροφου, υπολογίσθηκαν όλα τα γινόμενα $ 809 * i $ όπου $i$ παίρνει τιμές από 1 έως 1000. Βρέθηκε πως ο αντίστροφος είναι το 464.

\subsection*{({\lt iii})}
Καθώς το $2^{100}$ είναι δύσκολο να αποθηκευτεί και να χρησιμοποιηθεί σε πράξεις με ακρίβεια, χρησιμοποιήθηκε μία διαφορετική τεχνική. Υπολογίσθηκε το 2 {\lt modulo} 101 και το αποτέλεσμα πολλαπλασιάστηκε({\lt modulo} 101) με το 2. Αυτό έγινε επαναληπτικά 100 φορές και το αποτέλεσμα είναι 464.

\subsection*{({\lt iv})}
Ο αλγόριθμος υλοποιήθηκε στο αρχείο {\lt fast.py}. Τα αποτελέσματα είναι:
\[2^{1234567} \textrm{\lt mod} 12345 = 8648\]
\[130^{7654321} \textrm{\lt mod} 567 = 319\]
\newpage


% ===== Θέμα 3 =====
\section*{Θέμα 3}

\newpage


% ===== Θέμα 4 =====
\section*{Θέμα 4}

% ===== Θέμα 5 =====
\section*{Θέμα 5}


\newpage
% ===== Θέμα 6 =====
\section*{Θέμα 6}
\subsection*{({\lt i})}
\[{123454 \over 546542} = {0 + {1 \over {4 + {52726 \over 123454}}}}\]
\[ = {0 + {1 \over {4 + {1 \over {2 + {18002 \over 52726}}}}}}} = {0 + {1 \over {4 + {1 \over {2 + { 1 \over {2 + {16722 \over 18002}}}}}}}}} = {0 + {1 \over {4 + {1 \over {2 + { 1 \over {2 + {1 \over {1 + {1280 \over 16722}}}}}}}}}}}\]
\[ = {0 + {1 \over {4 + {1 \over {2 + { 1 \over {2 + {1 \over {1 + {1 \over {13 + {82 \over 1280}}}}}}}}}}}}} = {0 + {1 \over {4 + {1 \over {2 + { 1 \over {2 + {1 \over {1 + {1 \over {13 + {1 \over {15 + {50 \over 82}}}}}}}}}}}}}}}} = {0 + {1 \over {4 + {1 \over {2 + { 1 \over {2 + {1 \over {1 + {1 \over {13 + {1 \over {15 + {1 \over {1+ {32 \over 50}}}}}}}}}}}}}}}}}}\]
\[ = {0 + {1 \over {4 + {1 \over {2 + { 1 \over {2 + {1 \over {1 + {1 \over {13 + {1 \over {15 + {1 \over {1+ {1 \over {1 + {18 \over 32}}}}}}}}}}}}}}}}}}}} = {0 + {1 \over {4 + {1 \over {2 + { 1 \over {2 + {1 \over {1 + {1 \over {13 + {1 \over {15 + {1 \over {1+ {1 \over {1 + {1 \over {1 + {14\over 18}}}}}}}}}}}}}}}}}}}}}} \]
\[ = {0 + {1 \over {4 + {1 \over {2 + { 1 \over {2 + {1 \over {1 + {1 \over {13 + {1 \over {15 + {1 \over {1+ {1 \over {1 + {1 \over {1 + {1 \over {1+ {4 \over 14}}}}}}}}}}}}}}}}}}}}}}}} = {0 + {1 \over {4 + {1 \over {2 + { 1 \over {2 + {1 \over {1 + {1 \over {13 + {1 \over {15 + {1 \over {1+ {1 \over {1 + {1 \over {1 + {1 \over {1+ {1 \over {3 + {2\over 4}}}}}}}}}}}}}}}}}}}}}}}}}} \]
\[ = {0 + {1 \over {4 + {1 \over {2 + { 1 \over {2 + {1 \over {1 + {1 \over {13 + {1 \over {15 + {1 \over {1+ {1 \over {1 + {1 \over {1 + {1 \over {1+ {1 \over {3 + {1\over {2 + 0}}}}}}}}}}}}}}}}}}}}}}}}}}} \]
\newpage


% ===== Θέμα 7 =====
\section*{Θέμα 7}

% ===== Θέμα 8 =====
\section*{Θέμα 8}

\newpage


% ===== Θέμα 9 =====
\section*{Θέμα 9}
\subsection*{(α)} Εφαρμόστηκε το test του {\lt Fermat} με τις συνθήκες που αναφέρονται στην εκφώνηση. Αυτό που παρατηρείται είναι ότι, κατά προσέγγιση, οι μισοί αριθμοί που παράγονται από την $f(x)$ είναι πρώτοι ενώ οι υπόλοιποι μισοί όχι.
\subsection*{(β)} Τα μόνα πολυώνυμα του $\mathbb{Z}$ που, για κάθε ακέραια τιμή, δίνουν πρώτο αριθμό είναι τα σταθερά πολυώνυμα με τιμή κάποιον πρώτο αριθμό.
\[P(x) = p\] όπου $p$ πρώτος αριθμός

Αντιθέτως, δεν υπάρχει μή σταθερό πολυώνυμο για το οποίο να ισχύει η παραπάνω συνθήκη.\\

\underline{Απόδειξη}

Έστω ένα μη σταθερό πολυώνυμο $P(x)$ που μας δίνει πρώτο αριθμό για κάθε τιμή του $x \in \mathbb{Z}$

Τότε ισχύει
\[P(1) = p\]
όπου p πρώτος αριθμός.

Επομένως ισχύει και
\[P(1 + np) \equiv 0 ( \textrm{\lt mod} p), n \in \mathbb{Z} \]
γιατί η παραπάνω τιμή του $P$ διαιρείται με το $p$. Επομένως, αφού η τιμή του $P(1+np)$ διαιρείται με το $p$, δεν είναι πρώτος αριθμός. Άτοπο, άρα η αρχική υπόθεση είναι εσφαλμένη και δεν υπάρχει τέτοιο μη σταθερό πολυώνυμο.

\end{document}
