% Θεμελιώσεις Κρυπτογραφίας 2016
% Εργασία #2
% Κωσταντίνος Σαΐτας - Ζαρκιάς - 2406
% Οδυσσεύς Κρυσταλάκος - 2362
%-------------------------------------------------------------------------

\documentclass[a4paper, 11pt]{article}


\usepackage[english,greek]{babel} % the last language is the default
	\usepackage[utf8x]{inputenc}

%% > UNCOMMENT if your editor uses iso-8859-7 encoding for Greek (typical in Windows System).
% \usepackage[iso-8859-7]{inputenc}

\usepackage{enumerate}
\usepackage{seqsplit}
\usepackage{hyperref}
\usepackage[pdftex]{graphicx}


\newcommand{\lt}{\latintext}
\newcommand{\gt}{\greektext}
\newcommand\tab[1][1cm]{\hspace*{#1}}
%-------------------------------------------------------------------------

\title{Εργασία 2}

\author{Κωσταντίνος Σαΐτας - Ζαρκιάς - 2406 \\ Οδυσσεύς Κρυσταλάκος - 2362}

\date{\today}

%--------------------------------------------------------------------------
\begin{document}

\maketitle

% ===== Θέμα 1 =====
\section*{Θέμα 1}


\subsection*{({\lt i})}

Το πρωτόκολλο {\lt Diffie - Hellman} είναι μια μέθοδος ανταλλαγής κλειδιού που βασίζεται σε μια αμφίδρομη μέθοδο δημιουργίας ενός κλειδιού μεταξύ 2 ατόμων.

Η μέθοδος μπορεί να αναλυθεί στα εξής παρακάτω βήματα:

\begin{itemize}

\item[1 -] Εύρεση 2 πρώτων αριθμών {\lt p} και {\lt g} από την μία πλευρά και μετάδοση τους στο άλλο άκρο.

\item[2 -] Η μια πλευρά επιλέγει έναν μυστικό αριθμό {\lt a}, υπολογίζει την παράσταση $A = g^a mod p$ και αποστέλλει στην άλλη πλευρά μόνο το αποτέλεσμα {\lt A}. Αντίστοιχα, η άλλη πλευρά κάνει την ίδια διαδικασία με έναν μυστικό αριθμό {\lt b} και αποστέλλει το αποτέλεσμα {\lt B}.

\item[3 -] Υπολογισμός της παράστασης $ K1 = B^a mod p $ από την μια πλευρά και της παράστασης $ K2 = A^b mod p $ από την άλλη.

\item[4 -] Παρατηρούμε όμως ότι Κ1 = Κ2 καθώς $ (g^a mod p)^b mod p = (g^b mod p)^a mod p $ και συνεπώς και οι 2 πλευρές έχουν το κλειδί.


\end{itemize}

Η μέθοδος αυτή βασίζεται στο πρόβλημα του διακριτού λογάριθμου που είναι υποεκθετικού χρόνου και πολύ δύσκολο στην επίλυσή του αν οι αριθμοί που επιλεχθούν είναι κατάλληλα μεγάλοι.


\subsection*{({\lt ii})} Ένα σύστημα επικοινωνίας στο οποίο δημοσιεύεται ένα δημόσιο κλειδί σε μη-ασφαλή δίαυλο και μέσω αυτού οι 2 πλευρές δημιουργούν ένα κοινό μυστικό κλειδί ονομάζεται κρυπτοσύστημα δημοσίου κλειδιού (ασυμμετρική κρυπτογράφηση) και εφευρέθηκε από τους {\lt Diffie} και {\lt Hellman}. Με αυτήν την μέθοδο δεν γίνεται ποτέ ανταλλαγή των ίδιων των μυστικών κλειδιών μέσω του διαύλου επικοινωνίας και συνεπώς δεν υπάρχει ο φόβος υποκλοπής του μυστικού κλειδιού	.


\subsection*{({\lt iii})} Μια συνάρτηση $ f $ ονομάζεται {lt trapdoor function} αν είναι εύκολο να υπολογίσουμε το $ y = f(x) $ αλλά δύσκολο την $ x = f^-1(y) $ χωρίς κάποια επιπλέον πληροφορία {\lt k}. Με απλά λόγια, μια {\lt trapdoor function} είναι μια είδους συνάρτηση η οποία είναι πολύ εύκολο να υπολογιστεί από την μια φορά αλλά πολύ δύσκολο να υπολογιστεί από την αντίθετη αν δεν υπάρχει καμία επιπλέον πληροφορία.


\subsection*{({\lt iv})} Το πρωτόκολλο ανταλλαγής κλειδιού {\lt Diffie - Hellman} από μόνο του δεν προσφέρει αυθεντικοποίηση των χρηστών και συνεπώς είναι ευάλωτο σε επιθέσεις {\lt Man-In-The-Middle}. Μια λύση για το πρόβλημα αυτό είναι η χρήση ψηφιακών πιστοποιητικών ({\lt Digital Certificates}) τα οποία εξασφαλίζουν στον δέκτη ότι ο πομπός ήταν όντως αυτός που υποστηρίζει ότι είναι μέσω ενός έμπιστου τρίτου προσώπου. Συνεπώς, ο {\lt Bob} και η {\lt Alice} χρησιμοποιώντας ο καθένας την δικιά του ψηφιακή υπογραφή μπορούν να εξασφαλίσουν ότι το κλείδι που αντάλλαξαν προήλθε πραγματικά από τους ίδιους χωρίς την εμπλοκή της {\lt Eve}.


\subsection*{({\lt v})} Η εντροπία κατα {\lt Shannon} αφόρα την αναμενόμενη τιμή ενός μηνύματος σε ένα σύστημα. Συγκεκριμένα, πόση πληροφορία προσφέρει ένα σύστημα με την έξοδό του αναλόγως την πιθανότητα εύρεσης της αναμενόμενης τιμής των πιθανών μηνυμάτων. Στο παράδειγμα της ρίψης ενός νομίσματος που το αποτέλεσμα είναι 50/50 η πληροφορία που εξάγεται από την έξοδο του συστήματος είναι μικρή, δηλαδή 1 {\lt shannon} που προκύπτει από τον αριθμό των {\lt bits} που χρειάζονται για να αναπαραστίσουν τις πιθανές εξόδους του συστήματος όταν η πιθανότητα αποτελέσματος για όλες τις εξόδους είναι ίδια για όλες.


\subsection*{({\lt vi})}
Υπάρχουν 4 τύποι επιθέσεων στις ψηφιακές υπογραφές:
\begin{itemize}
	\item {\lt Existential forgery}: Ο επιτιθέμενος μπορεί να παράξει υπογραφεί για κάποιο μήνυμα $m$ στο οποίο δεν έχει καμία επιρροή. Αυτό σημαίνει ότι ο επιτηθέμενος δεν διαλέγει το $m$ και το μήνυμα δεν είναι απαραίτητο να έχει κάποιο νόημα.
	\item {\lt Selective forgery}: Ο επιτηθέμενος επιλέγει ένα μήνυμα $m$ και στην συνέχεια μπορεί να παράξει υπογραφή για αυτό το συγκεκριμένο $m$.
	\item {\lt Universal forgery}: Ο επιτηθέμενος μπορεί να παράξει ψηφιακή υπογραφή για οποιοδήποτε μήνμυμα $m$
	\item {\lt Total break}: Ο επιτηθέμενος αποκτά πρόσβαση στο ιδιωτικό κλειδί.
\end{itemize}

\subsection*{({\lt vii})}
Με $p$ και $q$ γνωστά μπορούμε να υπολογίσουμε το $N$ και $φ(N)$.
\[N = p \cdot q = 463 \cdot 547 = 253261\]
\[\phi(N) = (p-1)\cdot(q-1) = 462 \cdot 546 = 252252\]

επομένως το $d$ υπολογίζεται:
\[ed\equiv1(\textrm{\lt mod} \phi(N)) \Leftrightarrow d = 27473 \]

Τελικά για την αποκρυπτογράφηση του c αρκεί να υπολογισθεί
\[ c^d (\textrm{\lt mod} N) = 12584\]

\subsection*{({\lt vii})}
Ο λόγος που χρησιμοποιείται μία συνάρτηση κατακερματισμού πριν την υπογραφή του μηνύματος είναι η αποφυγή {\lt existential forgergeries}. Για παραδειγμα, έστω το δημόσιο κλειδί $(y, N)$. Κάποιος θα μπορούσε να παράξει μία τυχαία υπογραφή, έστω $s$. Αυτή η υπογραφή αντιστοιχεί στο μήνυμα $m = s^y$ και περνά τον έλεγχο. Επομένως ο επιτηθέμενος κατάφερε να υπογράψει ένα τυχαίο μήνυμα $m$ δηλαδή κατάφερε {\lt existential forgery}. Για να αποφευχθεί αυτό το πρόβλημα, το μηνύματα περνούν από μία {\lt hash} $Η$. Έτσι είναι δύσκολο να επιλεχθεί $s$ ώστε $Η(m) = s^y$.

\subsection*{({\lt ix})}
Έστω ότι έχουμε σύστημα ψηφιακής υπογραφής με τα εξής στοιχεία: \\
$H$: Συνάρτηση κατακερματισμού\\
$p$: Μεγάλος πρώτος ακέραιος\\
$g$: Τυχαίος {\lt generator} στην κυκλική ομάδα $Z^*_p$\\
$x$: Μυστικό κλειδί $< p$\\
$y$: Δημόσιο κλειδί με $y=g^x\textrm{\lt mod}p}$\\

Έστω πώς η {\lt Alice} θέλει να υπογράψει ένα μήνυμα $m$ και να το στείλει στον {\lt Bob}. Κατά τη διαδικασία της υπογραφής, επιλέγεται ένα τυχαίο $k$ έτσι ώστε $ 1 < k < p-1 $. Επίσης υπολογίζονται:
\[ r \equiv g^k \pmod{p} \]
\[ s \equiv k^{-1}(H(m)-xr) \pmod{p-1}\]
Έτσι στον {\lt Bob} αποστέλονται τα $(m,r,s)$. Φυσικά αυτά μπορεί να τα λάβει και οποιοσδήποτε άλλος παρακολουθεί το κανάλι επικοινωνίας.

Έστω ότι η {\lt Alice} χρησιμοποιεί δύο φορές το ίδιο $k$. Τότε για δύο διαφορετικά μηνύματα $m_1$ και $m_2$ ισχύει:
\[ s_1 \equiv k^{-1}(H(m_1)-xr) \pmod{ p-1}\]
\[ s_2 \equiv k^{-1}(H(m_2)-xr) \pmod{ p-1}\]
εκ των οποίων προκύπτει:
\[k(s_2 - s_1) = H(m_2) - H(m_1)\]
Από αυτή την εξίσωση μπορούν να βρεθούν μία σειρά από $k$ τα οποία αν αντικατασταθούν στην σχέση:
\[ r \equiv g^k \pmod{ p} \]
μπορεί να βρεθεί η μοναδική σωστή τιμή του $k$. Με αντικατάσταση στις παραπάνω σχέσεις, μπορεί να βρεθεί το ιδιωτικό κλειδί.

Τελικά, o {\lt Bob} ή οποιοσδήποτε άλλος παρακολουθεί το κανάλι μπορεί να υπογράφει με την υπογραφή της {\lt Alice}.

\subsection*{({\lt x})}
Ισχύει
\[\phi(n) = (p-1)(q-1) = pq - p - q + 1 = pq - (p+q) + 1\]
και επειδή $N = pq$
\[\phi(n) = N - (p+q) + 1\]
\newpage


% ===== Θέμα 2 =====
\section*{Θέμα 2}
\subsection*{({\lt i})}
Υλοποιώντας και χρησιμοποιώντας τον εκτεταμένο αλγόριθμο του Ευκλείδη για την εύρεση του {\lt GCD}, βρέθηκε:

\[GCD(126048, 5050) = 202\]
\[-1 \cdot 126048 + 25 \cdot 5050 = 202\]

\subsection*{({\lt ii})}
Για τον υπολογισμό του αντίστροφου, υπολογίσθηκαν όλα τα γινόμενα $ 809 * i $ όπου $i$ παίρνει τιμές από 1 έως 1000. Βρέθηκε πως ο αντίστροφος είναι το 464.

\subsection*{({\lt iii})}
Καθώς το $2^{100}$ είναι δύσκολο να αποθηκευτεί και να χρησιμοποιηθεί σε πράξεις με ακρίβεια, χρησιμοποιήθηκε μία διαφορετική τεχνική. Υπολογίσθηκε το 2 {\lt modulo} 101 και το αποτέλεσμα πολλαπλασιάστηκε({\lt modulo} 101) με το 2. Αυτό έγινε επαναληπτικά 100 φορές και το αποτέλεσμα είναι 464.

\subsection*{({\lt iv})}
Ο αλγόριθμος υλοποιήθηκε στο αρχείο {\lt fast.py}. Τα αποτελέσματα είναι:
\[2^{1234567} \textrm{\lt mod} 12345 = 8648\]
\[130^{7654321} \textrm{\lt mod} 567 = 319\]
\newpage


% ===== Θέμα 3 =====
\section*{Θέμα 3}
\subsection*{({\lt i})}
Ισχύει $ed \equiv 1 \pmod{1} $. Δοκιμάζοντας κάθε πιθανό $d$ για $1 < d < \phi(n)$ βρέθηκε $d = 2532979$.

\subsection*{({\lt ii})}
Για την εύρεση των $p$,$q$ γνωρίζουμε δύο σχέσεις:
\begin{equation}
N = pq
\end{equation}

\begin{equation}
N + 1 - \phi(N) = p + q
\end{equation}

\noindent Για την επίλυση του συστήματος:
\begin{equation}
(1) \Rightarrow p = N \over q
\end{equation}

\begin{equation*}
(2),(3) \Rightarrow N + 1 - \phi(N) = {N\over q} + q \Rightarrow qN + q - q\phi(N) = N + q^2
\end{equation*}
\[ \Rightarrow q^2 - (N + 1 - \phi(N))q + N = 0 \]

\noindent Λύνοντας την δευτεροβάθμια, προκύπτει ότι το $q$ πάιρνει τις τιμές 1999 και 2003. \\

Επομένως $p = 1999$ και $q=2003$ ή το αντίστροφο.

\newpage


% ===== Θέμα 4 =====
\section*{Θέμα 4}


Για την επίλυση των παρακάτω γραμμικών ισοδυναμιών εκτελέστηκε η εξής μέθοδος που βασίζεται στο Κινέζικο Θεώρημα Υπολοίπων.

$ x = 9 (mod 17)$
$ x = 9 (mod 12)$
$ x = 13 (mod 19)$

Έστω $ m1 = 17, m2 = 12, m3 = 19$ και $ n1 = 9, n2 = 9, n3 = 13 $.

\begin{enumerate}

\item Έλεγχος αν οι μεταβλητές {\lt m1,m2,m3} είναι πρώτοι μεταξύ τους χρησιμοποιώντας τον αλγόριθμο του ευκλείδη, δηλαδή να έχουν μέγιστο κοινό διαιρέτη το 1. Εφόσον περάσουν τον έλεγχο υπολογίζεται η λύση του συστήματος με το Κινέζικο Θεώρημα Υπολοίπων.

\item Έστω $ M = m1 * m2 * m3 $ και $ M_i = M / m_i $ για $ i = 1,2,3 $ αντίστοιχα.

\item Υπολογισμός των αντίστροφων στοιχείων $ u_i $ των εξισώσεων $ u_i * M_i = 1 (mod m_i) $ με την συνάρτηση {\lt modular inverse}.

\item Υπολογισμός του $ x = \sum_{i=1}^{3} n_i * u_i * M_i $

\item Η απάντηση είναι $ x mod M $ και πρέπει να είναι μεταξύ 0 και 1000.

Συνεπώς, το αποτέλεσμα είναι το 621.

\end{enumerate}

% ===== Θέμα 5 =====
\section*{Θέμα 5}


Για την αποκρυπτογράφηση του δοσμένου κρυπτογραφημένου μηνύματος με {\lt textbook RSA} καθώς τα {\lt N} και {\lt e} είναι γνωστά έπρεπε πρώτα να βρεθεί το ιδιωτικό κλειδί και ύστερα εύρεση των τιμών {\lt ASCII} δουλεύοντας {\lt block by block} στο {\lt C}. \\

Για την εύρεση του ιδιωτικού κλειδιού εκτελέστηκαν τα εξής βήματα:

\begin{enumerate}

\item Εύρεση των πρώτων παραγόντων {\lt p, q} του {\lt N}. Εύρεση του ενός πρώτου παράγοντα με {\lt brute force} και ύστερα διαίρεση το {\lt N} με τον αριθμό αυτό για την εύρεση του άλλου παράγοντα.

\item Εύρεση του φ(Ν) από τον τύπο: φ(Ν) $ = (p - 1) * (q - 1) $

\item Εύρεση του μυστικού κλειδιού {\lt d} από τον τύπο: $ e * d = 1 mod $ φ(Ν) με την συνάρτηση {\lt modular inverse}

\item Αποκρυπτογράφηση του κάθε {\lt block} ξεχωριστά με το {\lt d } και το {\lt N}  με την συνάρτηση {\lt fast} και μετατροπή της τιμής {\lt ASCII} σε γράμμα.

\end{enumerate}

Το αποκρυπτογραφημένο μήνυμα είναι: "{\lt welcowe to real world}"

\newpage
% ===== Θέμα 6 =====
\section*{Θέμα 6}
\subsection*{({\lt i})}
\[{123454 \over 546542} = {0 + {1 \over {4 + {52726 \over 123454}}}}\]
\[ = {0 + {1 \over {4 + {1 \over {2 + {18002 \over 52726}}}}}} = {0 + {1 \over {4 + {1 \over {2 + { 1 \over {2 + {16722 \over 18002}}}}}}}} = {0 + {1 \over {4 + {1 \over {2 + { 1 \over {2 + {1 \over {1 + {1280 \over 16722}}}}}}}}}}\]
\[ = {0 + {1 \over {4 + {1 \over {2 + { 1 \over {2 + {1 \over {1 + {1 \over {13 + {82 \over 1280}}}}}}}}}}}}} = {0 + {1 \over {4 + {1 \over {2 + { 1 \over {2 + {1 \over {1 + {1 \over {13 + {1 \over {15 + {50 \over 82}}}}}}}}}}}}}}}} = {0 + {1 \over {4 + {1 \over {2 + { 1 \over {2 + {1 \over {1 + {1 \over {13 + {1 \over {15 + {1 \over {1+ {32 \over 50}}}}}}}}}}}}}}}}}}\]
\[ = {0 + {1 \over {4 + {1 \over {2 + { 1 \over {2 + {1 \over {1 + {1 \over {13 + {1 \over {15 + {1 \over {1+ {1 \over {1 + {18 \over 32}}}}}}}}}}}}}}}}}} = {0 + {1 \over {4 + {1 \over {2 + { 1 \over {2 + {1 \over {1 + {1 \over {13 + {1 \over {15 + {1 \over {1+ {1 \over {1 + {1 \over {1 + {14\over 18}}}}}}}}}}}}}}}}}}}} \]
\[ = {0 + {1 \over {4 + {1 \over {2 + { 1 \over {2 + {1 \over {1 + {1 \over {13 + {1 \over {15 + {1 \over {1+ {1 \over {1 + {1 \over {1 + {1 \over {1+ {4 \over 14}}}}}}}}}}}}}}}}}}}}}} = {0 + {1 \over {4 + {1 \over {2 + { 1 \over {2 + {1 \over {1 + {1 \over {13 + {1 \over {15 + {1 \over {1+ {1 \over {1 + {1 \over {1 + {1 \over {1+ {1 \over {3 + {2\over 4}}}}}}}}}}}}}}}}}}}}}}}} \]
\[ = {0 + {1 \over {4 + {1 \over {2 + { 1 \over {2 + {1 \over {1 + {1 \over {13 + {1 \over {15 + {1 \over {1+ {1 \over {1 + {1 \over {1 + {1 \over {1+ {1 \over {3 + {1\over {2 + 0}}}}}}}}}}}}}}}}}}}}}}}}} \]
\newpage

\subsection*{({\lt ii})}
Εφαρμόστηκε η επίθεση του {\lt Wiener} όπως περιγράφεται στις σημειώσεις και στον ψευδοκώδικα. Η μόνο αλλαγή που έγινε ήταν η χρήση ακέραιας διαίρεσης κατά την εύρεση του $\phi$. Επίσης χρείαστηκε να υλοποιηθούν οι αλγόριθμοι για την εύρεση του συνεχούς κλάσματος. Τελικά, υπολογίσθηκαν δύο πιθανές τιμές για το $d$ οι οποίες είναι 3 και 20881. Καθώς η τιμή 3 δεν οδηγεί σε αριθμούς εντός των ορίων του {\lt ASCII}, απορρίπτεται. Η τιμή 20881 οδήγησε στο σωστό κείμενο που είναι: \\

{\lt  Just because you are a character doesn't mean that you have character}
% ===== Θέμα 7 =====
\section*{Θέμα 7}


Η εύρεση του μηνύματος που κρυπτογραφήθηκε με την εφαρμογή της {\lt trapdoor function} του {\lt Rabin} έγινε με την βοήθεια του Κινέζικου Θεωρήματος Υπολοίπων ως αναπαράσταση του προβλήματος σε σύστημα γραμμικών ισοδυναμιών. Καθώς, οι πρώτοι παράγοντες $p, q$ είναι γνωστοί γίνεται να υπολογιστεί η αντίστροφη συνάρτηση $ F^-1(sk, y) $ για την εύρεση του $x$.

\begin{enumerate}

\item Έστω $ p = 5, q = 11, N = p * q, c = 14 (=y) $. Από την συνάρτηση
\[ F^-1(sk,y): x^2 = y (mod N) \Leftrightarrow x^ 2 = 14 (mod 55) \Leftrightarrow \]
\[ \Leftrightarrow \left\{
	\begin{array}{ll}
		x^2 = 14 (mod 5) \Leftrightarrow x = 2\\
		x^2 = 14 (mod 11) \Leftrightarrow x = 5
	\end{array}
\right.\]
βρέθηκαν τα $ x = 2 $ και $ x =5 $ με την χρήση {\lt brute force}.

\item Συνεπώς, προκύπτουν τα εξής 4 συστήματα: \\
$ x = y_p (mod p) = 2 (mod 5) \\ $
$ x = -y_p (mod p) = -2 (mod 5) \\ $
$ x = y_q (mod q) = 5 (mod 11) \\ $
$ x = -y_q (mod q) = -5 (mod 11)$

\item Για την επίλυση των συστημάτων χρησιμοποιήθηκε το {\lt CRT} με τις αντίστοιχες μεταβλητές: $ m1 = p, m2 = q, n1 = 2 , n2 = 5, n3 = -2, n4 = -5 $ δίνοντας τις εξής 4 λύσεις: [27, 38, 17, 28].

\item Κατα την εκφώνηση, η λύση που είναι μικρότερη του 20 είναι η σωστή, δηλαδή το 17.

\end{enumerate}
\newpage

% ===== Θέμα 8 =====
\section*{Θέμα 8}


Από την άσκηση αυτή φαίνεται ο λόγος για τον οποίο όταν χρησιμοποιείται η κρυπτογράφηση του {\lt RSA} δεν πρέπει να γίνεται ποτέ μόνη της, χωρίς {\lt pad}, και να μην χρησιμοποιείται ο ίδιος εκθέτης {\lt e}. Στο συγκεκριμένο παράδειγμα, είναι γνωστό πως το μήνυμα {\lt m} έχει κρυπτογραφηθεί 3 φορές με τον ίδιο εκθέτη $ e = 3 $ και τα αντίστοιχα $ N[i] = 391, 55, 87 $ και $ c[i] = 208, 38, 32 $. Καθώς, το {\lt e} είναι γνωστό και επίσης είναι γνωστό ότι και στις 3 εξισώσεις το αποτέλεσμα ({\lt m}) είναι κοινό αρκεί να λυθεί το εξής σύστημα: $ m^3 = c[i] mod N[i] , i = 1,2,3 $ , δηλαδή:

\[ x = 208 \pmod{391} \\ \]
\[ x = 38 \pmod{55}  \\ \]
\[ x = 32 \pmod{87}  \\ \]

Ένα τέτοιο σύστημα, όπως έχει υποθεί και προηγουμένος, λύνεται εύκολα με το Κινέζικο Θεώρημα Υπολοίπων. Συνεπώς, χρησιμοποιώντας την ήδη υλοποιημένη συνάρτηση {\lt CRT} η λύση του συστήματος είναι $ x = 103823 $ και καθώς το μήνυμα έχει εκθέτη το 3, η απάντηση είναι η τρίτη ρίζα του $x$, δηλαδή το 47.


\newpage


% ===== Θέμα 9 =====
\section*{Θέμα 9}
\subsection*{(α)} Εφαρμόστηκε το test του {\lt Fermat} με τις συνθήκες που αναφέρονται στην εκφώνηση. Αυτό που παρατηρείται είναι ότι, κατά προσέγγιση, οι μισοί αριθμοί που παράγονται από την $f(x)$ είναι πρώτοι ενώ οι υπόλοιποι μισοί όχι.
\subsection*{(β)} Τα μόνα πολυώνυμα του $\mathbb{Z}$ που, για κάθε ακέραια τιμή, δίνουν πρώτο αριθμό είναι τα σταθερά πολυώνυμα με τιμή κάποιον πρώτο αριθμό.
\[P(x) = p\] όπου {\lt $p$} πρώτος αριθμός

Αντιθέτως, δεν υπάρχει μή σταθερό πολυώνυμο για το οποίο να ισχύει η παραπάνω συνθήκη.\\

\underline{Απόδειξη}

Έστω ένα μη σταθερό πολυώνυμο $P(x)$ που μας δίνει πρώτο αριθμό για κάθε τιμή του $x \in \mathbb{Z}$

Τότε ισχύει
\[P(1) = p\]
όπου p πρώτος αριθμός.

Επομένως ισχύει και
\[P(1 + np) \equiv 0 \pmod{p}, n \in \mathbb{Z} \]
γιατί η παραπάνω τιμή του $P$ διαιρείται με το $p$. Επομένως, αφού η τιμή του $P(1+np)$ διαιρείται με το $p$, δεν είναι πρώτος αριθμός. Άτοπο, άρα η αρχική υπόθεση είναι εσφαλμένη και δεν υπάρχει τέτοιο μη σταθερό πολυώνυμο.

\section*{Θέμα 10}
\subsection*{(α)} Υλοποιήθηκε ο αλγόριθμος και ο πρώτος που παράγεται αποθηκεύεται στο αρχείο {\lt prime.txt}.
\subsection*{(β)} Με τη χρήση του αλγορίθμου του {\lt Shanks} βρέθηκε ότι ισχύουν τα εξής:
\[3^{27} = 2 \pmod{43}\]
\[3^{12} = 4 \pmod{43}\]
\[3^{25} = 5 \pmod{43}\]

Η μέθοδος {\lt Pollard-}ρ δεν μπόρεσε να δώσει έγκυρο αποτέλεσμα για αυτά τα δεδομένα (καθώς είναι πιθανοκρατικός αλγόριθμος).

\section*{Θέμα 11}
Αποστάλθηκαν {\lt e-mails} με τα δημόσια κλειδιά και τα κρυπτογραφημένα μηνύματα, και των δυο μας στις 19 Μαΐου.
\end{document}
